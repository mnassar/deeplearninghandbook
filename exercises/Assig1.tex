\newif\ifColor
% Add \Colortrue to settings.tex to use color
% Otherwise, will show a primarily grayscale version of the document for printing

\input{settings.tex}


\documentclass[11pt,oneside,a4paper]{article}
\usepackage{enumitem}

\usepackage{natbib}
\usepackage{breakcites} % Do Not let citations break out of the text frame
\usepackage{microtype} % Get rid of some frame busts automatically
% Note: if microtype causes error on ubuntu, run
% sudo apt-get install cm-super
\title{Assignment 1}
\author{CMPS 392}
\date{}
% \setcounter{tocdepth}{1}

\pdfobjcompresslevel=0

\usepackage{zref-abspage}

\setcounter{secnumdepth}{3} % Number subsubsections, because we reference them,
% so the reader needs numbers to find the correct place.


\usepackage[vcentering,dvips]{geometry}
\geometry{papersize={7in,9in},bottom=3pc,top=5pc,left=5pc,right=5pc,bmargin=4.5pc,footskip=18pt,headsep=25pt}


%%% Packages %%%
\usepackage{epsfig}
\usepackage{subfigure}
\usepackage[utf8]{inputenc}

% Needed for some foreign characters
\usepackage[T1]{fontenc}

\usepackage{amsmath}
\usepackage{subfigure}
\usepackage{amsfonts}
\usepackage{amsthm}
\usepackage{multirow}
\usepackage{colortbl}
\usepackage{booktabs}
% This allows us to cite chapters by name, which was useful for making the
% acknowledgements page
\usepackage{nameref}
% Make sure there is a space between the subsection number and subsection title
% in the table of contents.
% If we do not do this we end up with 2 digit subsection numbers colliding with
% the title.
% See https://tex.stackexchange.com/questions/7853/toc-text-numbers-alignment/7856#7856?newreg=d2632892dd0345f388619f12fa794b11
% \usepackage[tocindentauto]{tocstyle}
% \usetocstyle{standard}

\usepackage{bm}


\usepackage{float}
\newcommand{\boldindex}[1]{\textbf{\hyperpage{#1}}}
\usepackage{makeidx}\makeindex
% Make bibliography and index appear in table of contents
\usepackage[nottoc]{tocbibind}
% Using the caption package allows us to support captions that contain "itemize" environments.
% The font=small option makes the text of captions smaller than the main text.
\usepackage[font=small]{caption}

% Used to change header sizes
\usepackage{fancyhdr}



% \usepackage[chapter]{algorithm}
\usepackage{algorithmic}
% Include chapter number in algorithm number
% \renewcommand{\thealgorithm}{\arabic{chapter}.\arabic{algorithm}}


\theoremstyle{definition}
\newtheorem{example}{Example}[section]

% Define the P table cell environment
% It is the same as p, but centers the text horizontally
\usepackage{array}
\newcolumntype{P}[1]{>{\centering\arraybackslash}p{#1}}

% Rebuild the book document class's headers from scratch, but with different font size
% (this is for MIT Press style)
% Source: http://texblog.org/2012/10/09/changing-the-font-size-in-fancyhdr/
\newcommand{\changefont}{% 1st arg to fontsize is font size. 2nd arg is the baseline skip. both in points.
    \fontsize{9}{11}\selectfont
}
\fancyhf{}
\fancyhead[LE,RO]{\changefont \slshape \rightmark} %section
% \fancyhead[RE,LO]{\changefont \slshape \leftmark} %chapter
\fancyfoot[C]{\changefont \thepage} %footer
\pagestyle{fancy}
\input{math_commands.tex}

\usepackage[pdffitwindow=false,
pdfview=FitH,
pdfstartview=FitH,
pagebackref=true,
breaklinks=true,
\ifColor
colorlinks,
\fi
bookmarks=false,
plainpages=false]{hyperref}

% Make \[ \] math have equation numbers
\DeclareRobustCommand{\[}{\begin{equation}}
\DeclareRobustCommand{\]}{\end{equation}}

% Allow align environments to cross page boundaries.
% If we do not do this, we get weird gaps of several inches of white space
% before or after some long align environments.
\allowdisplaybreaks

\begin{document}

\setlength{\parskip}{0.25 \baselineskip}
\newlength{\figwidth}
\setlength{\figwidth}{26pc}
% Spacing between notation sections
\newlength{\notationgap}
\setlength{\notationgap}{1pc}

% \typeout{START_CHAPTER "TOC" \theabspage}
%\frontmatter


\maketitle
%\tableofcontents
%\typeout{END_CHAPTER "TOC" \theabspage}

%  \input{notation.tex}
 %\mainmatter
%  \input{commentary.tex}

\section*{Linear Algebra} 
\subsection*{Q1}
Prove that $\displaystyle \vx^\top \mB \vx = 0 $ if $\mB$ is the anti-symmetric part of a matrix $A$. 

\subsection*{Q2}
What is the gradient of $\mathrm{tr}(\mA)$: \\
$$ \nabla_{\mA} \mathrm{tr}(\mA)=?$$

\subsection*{Q3}
What is the gradient of the sigmoid function: 
$$\sigma(\vx) = \frac{1}{1+\exp(-\vw^\top \vx)}$$
$$ \nabla_{\vx} \sigma(\vx)=?$$

\subsection*{Q4} 
What is the following gradient: 
$$ \nabla_{\vx} \frac{\vx^\top \mA \vx}{|| \vx ||_2^2}$$
assuming $\mA$ is symmetric.

\subsection*{Q5}
Is $\mA^\top \mA$ always: 
\begin{itemize}
    \item positive definite ? 
    \item symmetric ?
    \item positive semi-definite ? 
\end{itemize}
Explain.


\subsection*{Q6}
Prove that: $$\mathrm{tr} (\mA\mB\mC) = \mathrm{tr} (\mC\mA\mB) $$


\subsection*{Q7}
We know that : 
\begin{itemize}
\item $\mathrm{tr}(\mA) = \sum_i \lambda_i $
\item $\mathrm{det}(\mA) =  \prod_i \lambda_i $
\end{itemize}

Prove it for the special case of a symmetric matrix $\mA$. 

\subsection*{Q8}
\begin{enumerate}[label=(\alph*)]
\item Is a diagonalizable matrix always non singular? 
\item Is a non-singular matrix always diagonalizable? 
\end{enumerate}

Give counter examples.

\subsection*{Q9}
What is the image of the unit circle under multiplication by the matrix: 
$$ \mA =  \left[ 
\begin{array}{cc} 
3 & 2 \\
2 & 4  
\end{array} \right]$$ 

Write $\mA$ under the form $\displaystyle \mA = \mU \mLambda \mU^\top$, then 
track how a point $\displaystyle \vx$ from the circle moves after multiplication 
by $\mU^\top$, then by $\mLambda$, and finally by $\mU$. 

\subsection*{Q10}
What is the SVD decomposition of the following matrix:

$$ \mA =  \left[ 
\begin{array}{cc} 
3 & 2 \\
2 & 4  \\ 
1 & 2 \\ 
\end{array} \right]$$
(write under the form : $ \mU \mSigma \mV^\top$ where $\mU$ and $\mV$ are orthogonal, and $\mSigma$ is diagonal). 

\section*{Programming Assignment} 
In this exercise, we will use linear regression to predict the weight of a fish based on five other attributes: $Length1$, $Length2$, $Length3$, $Height$ and $Width$ (You can also include $Species$ which is a categorical variable in the way you see convenient if you like): 
\begin{enumerate}
    \item  Download the data file "fish.csv" from Moodle. 
    \item The dataset has 159 datapoints. Let's divide them into 100 datapoints for training and 59 datapoints for testing. 
    \item Compute a linear regression model over the training data solely. 
    \item Compute the mean squared error (MSE) for the training data, and the MSE for the testing data, respectively.
\end{enumerate}

Next, we will try to increase the capacity of the model by enriching the representation of the data. 
Let's add $Length1^2$, $Length2^2$, $Length3^2$, $Height^2$ and $Width^2$ to the set of features of our model. Repeat the experiment and report the new training MSE and the new testing MSE. 

Next we can add $Length1^3$, $Length2^3$, $Length3^3$, $Height^3$ and $Width^3$ and so on. Each time you increase the capacity of your model compute the training MSE and the testing MSE. 

What are your observations? What is the best capacity for minimizing the testing MSE? What is the difference between optimization and machine learning? 




% \typen of out where {START_CHAPTER "bib" \theabspage}
% \bibliography{notation}
% \bibliographystyle{natbib}
% \clearpage
% \typeout{END_CHAPTER "bib" \theabspage}
% }
% \typeout{START_CHAPTER "index-" \theabspage}
% \printindex
% %\clearpage
% \typeout{END_CHAPTER "index-" \theabspage}
% %\newpage



\end{document}
